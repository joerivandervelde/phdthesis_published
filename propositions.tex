\documentclass[10pt]{article}
\usepackage[a5paper]{geometry}
\usepackage{mathpazo}
\renewcommand{\familydefault}{\sfdefault}

\begin{document}
\thispagestyle{empty}

\Large
Stellingen

\small

\begin{enumerate}

  \item Variant pathogenicity predictions can be biologically correct while still false from a clinical perspective. \textsl{(this thesis)}

  \item For some genes, evolutionary and structural properties seem to have no bearing on the estimated pathogenicity of variants, hinting at control by still unknown mechanisms. \textsl{(this thesis)}

  \item In variant interpretation "Too much emphasis on trying to prove pathogenicity, runs the risk of the Texas Sharpshooter Fallacy" (-- Gijs Santen\footnote{\tiny http://biosb.nl/wp-content/uploads/2015/09/Day-3-Santen-Clinical-collaboration.pdf}). This risk can be minimized by using false discovery estimates from healthy individuals. \textsl{(this thesis)}

  \item Knowing the specific strengths and weaknesses of a method is ultimately more important than its overall performance, however good, for understanding that method's true benefit. \textsl{(this thesis)}

  \item In-vivo variant pathogenicity assays based on worm genetics could represent a great trade-off between ethics, cost and relevance for human disease. \textsl{(this thesis)}

  \item Semantic web technology is surprisingly under used, especially considering how data integration is always of prime concern in life science projects. \textsl{(this thesis)}

  \item Generating web databases from blueprints requires you to think about the structure and meaning of data, which is its real advantage.

  \item Journals should only accept papers with quality codebases that have been tested and are comparable to lab experiments using controls.

  \item With medicine that can repair the genome, we will no longer need to be afraid of incidental findings.
  
  \item ...however, with this new medicine, there will be a completely new dilemma of what to do with 'likely pathogenic' variants as identified by genome diagnostics.
  
\end{enumerate}

\noindent
Propositions belonging to the doctoral thesis 'Translational software infrastructure for medical genetics', by K. Joeri van der Velde, 2017

\end{document}
